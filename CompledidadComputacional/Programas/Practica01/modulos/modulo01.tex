\section{Ruta más corta}

\subsection{Dar su forma canónica}

Problema de decisión: Dada una gráfica no dirigida $G = (V, E)$, donde $V$ es el conjunto de vértices y $E$ 
es el conjunto de aristas, y dados dos vértices $u$ y $v$, y un entero positivo $k$, ¿existe una trayectoria 
entre $u$ y $v$ cuyo peso total sea menor que $k$?


\subsection{Diseñar un algoritmo no-determinístico polinomial}

\begin{myitemize}
    \item Partimos que trabajaremos con:
    \begin{myitemize}
        \item Una grafica
        \item Aristas
        \item Vértices
        \item Dos vértices preseleccionados 
        \item un entero
    \end{myitemize}

    \item Vamos a comenzar con que distancias entre vértices son desconocidas. 
    \item Creamos un arreglo para guardar esas distancias (distancias) 
    \item Esas distancias se van a inicializar con valores infinitos 
    \item Excepto para la distancia del vertice de inicio y final la cual será 0.

    \item Usaremos una cola de prioridad para explorar todo los vértices. 
    \item El vertice de inicio será el primero en agregarse a esta cola con distancia 0 (elegido aleatoriamente)
    
    \item Para poder explorar toda la gráfica vamos a seleccionar los vértices adyacentes al vértice que 
    tengamos evaluando en la cola 
    \item Una vez alcanzado vamos a actualizar las distancias a sus vecinos 
    \item La actualización debe ser si es que encontramos un camino más corto
        
    \item Una vez que terminamos de revisar toda la grafica verificamos si la distancia entre los vértices de 
        inicio y final es menor a la distancia $k$
    \item Si lo es, hemos encontrado la ruta
    \item Si no, no hay una ruta menor.
    \item Nuestro proceso termina cuando todos los vértices fueron explorados o cuando la cola de prioridad está vacía.
\end{myitemize}

%%% nota de correcciones:  Ahora, en cada iteración del bucle principal, obtenemos aleatoriamente un vértice adyacente para explorar. Esto introduce no determinismo en la selección de vértices adyacentes y cumple con la primera fase del algoritmo no determinístico. 

\begin{verbatim}
    Función principal rutaMasCorta(grafica, inicio, final, k):
    // Paso 1: Inicialización
    distancias = inicializarDistancias(grafica, inicio)
    colaPrioridad = inicializarColaPrioridad(inicio)
    
    // Paso 2: Exploración
    mientras colaPrioridad no esté vacía:
        verticeActual = extraerVerticeMinimo(colaPrioridad)
        para cada vecino en grafica[verticeActual]:
            actualizarDistancia(distancias, verticeActual, vecino)
    
    // Paso 3: Verificación
    si distancia entre inicio y final < k:
        devolver "Sí, hay una ruta de peso menor que k entre inicio y final"
    sino:
        devolver "No, no hay una ruta de peso menor que k entre inicio y final"

Función inicializarDistancias(grafica, inicio):
    distancias = arreglo de tamaño igual al número de vértices en la gráfica
    para cada vértice en la gráfica:
        si vértice es igual a inicio:
            distancias[vértice] = 0
        sino:
            distancias[vértice] = infinito
    devolver distancias

Función inicializarColaPrioridad(inicio):
    colaPrioridad = cola de prioridad
    agregar inicio a colaPrioridad con prioridad 0
    devolver colaPrioridad

Función extraerVerticeMinimo(colaPrioridad):
    extraer y devolver vértice con menor prioridad de colaPrioridad

Función actualizarDistancia(distancias, verticeActual, vecino):
    si distancia desde inicio hasta vecino pasando por verticeActual es menor que distancias[vecino]:
        distancias[vecino] = distancia desde inicio hasta vecino pasando por verticeActual
        agregar vecino a colaPrioridad con prioridad distancias[vecino]

\end{verbatim}