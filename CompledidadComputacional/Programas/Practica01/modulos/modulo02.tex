\section{3-SAT}

\subsection{Dar su forma canónica}

\textbf{Representación:}

\begin{myitemize}
    \item Fórmula booleana: Es una expresión que combina variables verdaderas o falsas con operadores 
            lógicos $(AND, OR, NOT)$.
    \item Forma normal conjuntiva (CNF): Es una forma específica de expresar una fórmula booleana. 
    \item Tres literales: En el contexto de 3-SAT, cada variable en la fórmula CNF tiene exactamente tres 
            literales. Un literal es una variable verdadera o falsa.
\end{myitemize}

Por lo tanto, la forma canónica del problema 3-SAT es una pregunta: ¿Dada una fórmula booleana en CNF donde cada 
cláusula tiene exactamente tres literales, ¿existe una asignación de verdad (verdadero o falso para cada variable) 
que haga que toda la fórmula sea verdadera? Si tal asignación existe, decimos que la fórmula es satisfacible.

\subsection{Diseñar un algoritmo no-determinístico polinomial}

\begin{myitemize}
    \item Fase de adivinación:
    
    \textbf{Adivina la asignación de verdad:} Para cada variable en la fórmula, adivina si es verdadera o falsa. Esta es 
    la parte no determinística del algoritmo. No sabemos cuál es la correcta, al azar adivinamos. Dado que son dos 
    variables esto podría verse como que estamos lanzando una moneda. 

    \item Fase de verificación:
    
    \textbf{Verifica la asignación de verdad:} Ahora, con la asignación de verdad adivinada, verificamos si satisface 
    todas las cláusulas de la fórmula. Para hacer esto, recorremos cada cláusula y verificamos si al menos una variable 
    es verdadera bajo la asignación donde se adivina.
    \begin{myitemize}
        \item Si encontramos una cláusula que no es verdadera bajo la asignación adivinada, entonces sabemos que la 
                asignación adivinada no satisface la fórmula, y el algoritmo se detiene y devuelve un mensaje 
                de \textit{no encontrado}
        \item Si todas las cláusulas son verdaderas bajo la asignación adivinada, entonces la asignación adivinada 
        satisface la fórmula, y el algoritmo se detiene y devuelve \textit{si encontrado}
    \end{myitemize}
\end{myitemize}