% ------------------------------------------------------------------------------\
% -----------------------------------------------------------------------------\
% ----------------------------------------------------------------------------\
\section{Problema}
% ------------------------------------------------------------------------------\
% -----------------------------------------------------------------------------\
% ----------------------------------------------------------------------------\

10.3 COLOREACIÓN DE GRÁFICOS

Uno de los problemas clásicos de la teoría de grafos es el de colorear los vértices de un gráfico de 
tal manera que no se asigne el mismo color a dos vértices adyacentes (es decir, conectados por una arista). 
El número mínimo de colores necesarios para colorear G se llama número cromático, $\gamma$(G), de G.

En esta sección, mostraremos que este problema es NPC. El problema sigue siendo NPC incluso si todo lo que 
preguntamos es si $\gamma$(G) $\leq$ 3. Además, incluso si restringimos la pregunta al gráfico plano, el problema 
sigue siendo NPC. Incluso si restringimos el problema a una clase de gráficos planos con realización plana 
de buen comportamiento, el problema de si $\gamma$(G) $\leq$ 3 sigue siendo NPC. Uno de tales La definición para 
una realización con buen comportamiento es que todos los bordes son líneas rectas, ningún ángulo es inferior 
a 10 y las longitudes de los bordes están entre dos dados.
límites.

Primero consideramos el problema de las 3 colores, (3C), que se define de la siguiente manera:

Entrada: Un gráfico G(V, E).

Pregunta: ¿Se puede asignar un color a cada vértice, de modo que solo se utilicen tres colores y no se asigne 
el mismo color a dos vértices adyacentes? (En resumen: ¿es$\gamma$(G) $\leq$ 3?)

Teorema 10.6: 3C es NPC.


Prueba: Demostramos que 3SAT $ \alpha$ 3C. (La demostración de este teorema, y el siguiente, sigue los 
trabajos de Stockmeyer [2] y Garey, Johnson y Stockmeyer [3].) Sea el conjunto de literales, de la entrada I a 
3SAT, $\{x_{1 }, x_ {2}, . . . , X_{n}, \bar{X_{1}}, \bar{X_{2}}, ..., \bar{X_{n}} \}$ y las cláusulas 
sean $C_{1}$, $C_{2}$, ..., $C_{m}$.

El gráfico G(V. E), que es la entrada f(I) a 3C, se define de la siguiente manera:

$ V = \{ a,b \} \cup \{ x_{i}, \bar{x_{i}} \hspace{3mm} | \hspace{3mm} 1 \leq i \leq n \}  \cup \{ w_{ij} \hspace{3mm} | \hspace{3mm} 1 \leq i \leq 6 \text{y} 1 \leq j \leq m \}$

$ E = \{a - b\} \cup \{a - x_{i}, a - \bar{x_{i}}, x_{i} - \bar{x_{i}} \hspace{3mm} | \hspace{3mm} 1 \leq i \leq i \} \cup$\\
\textcolor{AzulRey}{.} \hspace{12mm} $\{w_{1j}, - w_{2j}, w_{1j} - w_{4j}, w_{2j} - w_{4j}, w_{4j} - w_{5j}, $ \\
\textcolor{AzulRey}{.} \hspace{12mm} $ w_{3j} - w_{5j}, w_{3j} - w_{6j}, w_{5j} - w_{6j}, w_{6j} - b \hspace{3mm} | \hspace{3mm} 1 \leq j \leq m \}  \hspace{3mm} \cup $ \\ 
\textcolor{AzulRey}{.} \hspace{12mm} $\{ \xi_{1j} -  w_{1j},  \xi_{2j} - w_{2j}, \xi_{3j} -  w_{3j}  \hspace{3mm} | \hspace{3mm} 1 \leq j \leq m  \hspace{4mm} \text{and} $ \\ 
\textcolor{AzulRey}{.} \hspace{12mm} $C_{j} = \{ \xi_{1j},  \xi_{2j},  \xi_{3j}, \}\} $\\ 


La estructura de las dos últimas partes de la definición de E, para cada j, se muestra en la Figura 10.2. El significado 
de esta estructura es el siguiente: Supongamos que cada uno de los vértices $\xi_{1j}, \xi_{2j}$ y $\xi_{3j}$ 
está coloreado 0 o 1 (asumimos que los tres colores son 0, 1 y 2) e ignoramos por el momento el vértice b. 
Afirmamos que $w_{6j}$, puede colorearse 1 o 2 si y solo si no los tres vértices $\xi_{1j}, \xi_{2j}$ y $\xi_{3j}$
están coloreados 0. Primero, es fácil ver que si $\xi_{1j}, \xi_{2j}$ y $\xi_{3j}$ están coloreados 0, 
entonces $w_{4j}$ también debe ser coloreado 0, y por lo tanto debemos ser coloreados 0. Pero, como el lector 
puede comprobar por sí mismo, si al menos uno de $\xi_{1j}, \xi_{2j}, \xi_{3j}$, está coloreado 1 entonces 
$w_{6j}$ se puede colorear 1. La estructura de las dos primeras partes de la definición de E se muestra en la 
figura 10.3. Claramente, si a tiene el color 2, entonces todos los vértices literales deben tener el color 0 y 1, 
uno de estos colores se usa para $w_{i}$ y el otro para $\bar{w_{i}}$. Supongamos que \textit{I} es satisfactoria 
mediante alguna asignación de valor de verdad a los literales. Para ver que \textit{f(I)} tiene 3 colores, asigne a 
a el color 2. Asigne al literal $\xi$ el color 1 si es 'verdadero' y 0 si es 'falso'. Ahora, dado que a ningún 
triple $\xi_{1j}, \xi_{2j}, \xi_{3j}$ se le asignan todos ceros, podemos colorear $w_{1i}, w_{2j}. . . w_{6j}$ de 
tal manera que $w_{6j}$ tenga el color 1, para todo $j = 1, 2, ..., m$. Por lo tanto, b es coloreable 0 y la 
coloración 3 de G está completa. Por el contrario, si G tiene 3 colores, llame al color de a 2 y al color de b 0. 
Claramente, todos los vértices literales están coloreados 0 y 1, y $w_{6j}$ no puede ser coloreado 0. Por lo tanto, 
para cada triple $\xi_{1j}, \xi_{2j}, \xi_{3j}$, no los tres están coloreados 0. Ahora, si asignamos un literal 
'verdadero' si y sólo si su vértice correspondiente está coloreado 1, la asignación satisface todas las cláusulas. \\ 


Antes de abordar el problema de la colorabilidad tridimensional de los gráficos planos en general, consideremos el 
gráfico plano D con forma de diamante de la Figura 10.4. Supongamos que lo coloreamos con los tres colores 0, 1 y 2. 
Si $u_{0}$ tiene el color 0, el circuito $u_{1} - u_{2} - u_{3} - u_{4} - u_ {1}$, tiene el color 1 y 2, 
alternativamente. Para definición, suponga que $ u_ {1} $ y $ u_ {3} $ están coloreados 1 y $u_ {2}$ y $ u_ {4}$ 
están coloreados 2. Ahora, hay dos colores posibles para nosotros; es decir, i.e 0 o 2. Si $u_ {5}$ tiene el color 0,
entonces $v_{1}$, tiene el color 2, $v_{4}$ tiene el color 1, $v_{6}$ tiene el color 0, $v_{2}$ tiene el color 2,
$u_{7}$ tiene el color 0, $v_{3}$ tiene el color 1 y $u_{8}$ tiene el color 0. Lo importante es el hecho es que 
$v_{1}$ y $v_{2}$ tienen el mismo color (2) y $v_{3}$ y $v_{4}$ tienen el mismo color (1). Si $u_{5}$ tiene el color 2, 
entonces $v_{1}$, tiene el color 0, $u_{8}$ tiene el color 1, $v_{3}$ tiene el color 0, $v_{7}$ tiene el color 2, 
$v_{2}$ tiene el color 0, $u_{6}$ tiene el color 1 y $v_{4}$ tiene el color 0; es decir, los cuatro vértices, 
$v_{1}, v_{2}, v_{3}$ y $v_{4}$ tienen colores idénticos. Concluimos que en cada coloración triple de D $v_{1}$ y $v_{2}$
deben tener el mismo color, $v_{3}$ y $v_{4}$ deben tener el mismo color, pero el color de $v_{1}$ y $v_{3}$ pueden ser 
o no el mismo. Así, D, efectivamente realiza un cruce de la coloración de $v_{1}$ a $v_{2}$, de $v_{3}$ a $v_{4}$ sin ninguna restricción en cuanto a
Se introduce la igualdad o desigualdad del color de $v_{1}$ y $v_{3}$.

El problema de las 3 colores de gráficos planos (3CP) se define de la siguiente manera:






















% ------------------------------------------------------------------------------\
% -----------------------------------------------------------------------------\
\subsection{Forma Canónica}
% ------------------------------------------------------------------------------\
% -----------------------------------------------------------------------------\


% ------------------------------------------------------------------------------\
% -----------------------------------------------------------------------------\
\subsection{Demostración}
% ------------------------------------------------------------------------------\
% -----------------------------------------------------------------------------\


% ------------------------------------------------------------------------------\
% -----------------------------------------------------------------------------\
\subsection{Demostrar transformación}
% ------------------------------------------------------------------------------\
% -----------------------------------------------------------------------------\


% ------------------------------------------------------------------------------\
% -----------------------------------------------------------------------------\
\subsection{Ejemplificar}
% ------------------------------------------------------------------------------\
% -----------------------------------------------------------------------------\


% ------------------------------------------------------------------------------\
% -----------------------------------------------------------------------------\
\subsection{Técnicas de demostración}
% ------------------------------------------------------------------------------\
% -----------------------------------------------------------------------------\


% ------------------------------------------------------------------------------\
% -----------------------------------------------------------------------------\
\subsection{Aplicación}
% ------------------------------------------------------------------------------\
% -----------------------------------------------------------------------------\