\section{3 colores son NP completos.}

Prueba: El problema de la coloración 3 pertenece a NP ya que podemos adivinar una coloración 3 y 
comprobar que es una coloración válida fácilmente en tiempo polinomial. Reducimos 3SAT al 
problema de los 3 colores. Esta es una prueba más complicada por dos razones. Primero, los dos
problemas tratan con objetos diferentes (expresiones booleanas versus gráficas). En segundo lugar, 
no podemos simplemente reemplazar un objeto (por ejemplo, vértice, arista) por otro (por 
ejemplo, cláusula); Tenemos que ocuparnos de toda la estructura. La idea es utilizar bloques de 
construcción y luego unirlos. Sea $E$ una instancia arbitraria de 3SAT. Tenemos que construir un 
gráfico $G$ tal que $E$ sea satisfactorio si y solo si G puede ser de 3 colores. Primero, 
construimos el triángulo principal M.\\ 

Como $M$ es un triángulo, requiere al menos tres colores. Etiquetamos $M$ con los "colores" $T$ 
(para verdadero), F (para falso) y A (vea el triángulo inferior en la figura 11.4). Estos colores son
usado sólo para la prueba; no son parte del gráfico. Posteriormente asociaremos estos colores con 
la asignación de valores de verdad a las variables de $E$. Para cada variable $x$, tenemos construye 
otro triángulo $M$, cuyos vértices están etiquetados como x, $\bar{x}$ y $A$, donde $A$ es el mismo 
vértice en $M$. Entonces, si hay $k$ variables, tendremos $k + 1$ triángulos, todos compartiendo un 
vértice común $A$ (ver figura 11.4). La idea es que, si $x$ está coloreado con el color $T$, entonces 
$\bar{x}$ debe colorearse con $F$ (ya que ambos están conectados a $A$), y viceversa. Esto es 
consistente con el significado de $\bar{x}$. \\ 

Ahora tenemos que imponer la condición de que al menos una variable en cada cláusula tenga el valor 1. 
Lo hacemos con la siguiente construcción. Supongamos que la cláusula es $(x + y + z)$. Introducimos 
seis nuevos vértices y los conectamos a los vértices existentes, como se muestra en la figura 11.5. 
(Las etiquetas son consistentes, de modo que solo hay un vértice en todo el gráfico etiquetado como 
$T$, y un vértice para cada $x, y,$ o $z$.) Llamemos a los tres nuevos vértices conectados a $T$ y 
$ x, y$ o $z$ los vértices externos (están etiquetados con O en la figura) y los tres nuevos vértices 
en el triángulo, los vértices internos (etiquetados con / en la figura). Afirmamos que esta 
construcción garantiza que, si no se utilizan más de 3 colores, entonces al menos uno de $x$, $y$ o $z$ 
debe tener el color $T$. Ninguno de ellos puede colorearse como $A$, ya que todos están conectados a $A$ 
(ver Fig. 11.4). Si todos están coloreados $F$, entonces los tres nuevos vértices conectados a ellos 
deben tener el color A, ¡pero entonces el triángulo interior no puede colorearse con tres colores! La 
gráfica completa correspondiente a la expresión $(\bar{x} + y + \bar{z}) \cdot (\bar{x} + \bar{y} + z)$\\ 

Ahora podemos completar la prueba. Tenemos que demostrar dos lados: (1) si $E$ es satisfactoria, entonces 
$G$ puede colorearse con tres colores; y (2) si $G$ se puede colorear con tres colores, entonces $E$ es 
satisfactorio. Si $E$ es satisfacible entonces hay una asignación de verdad satisfacible. Coloreamos los 
vértices asociados con las variables de acuerdo con esta asignación de verdad ($T$ si $x =1$, y F en 
caso contrario). $M$ está coloreado con $T$, $F$ y A como se indica. Cada cláusula debe tener al menos una 
variable cuyo valor sea 1. Por lo tanto, podemos colorear el vértice exterior correspondiente con $F$, el 
resto de los vértices exteriores con $A$ y el triángulo interior en consecuencia. Por tanto, $G$ se puede 
colorear con tres colores. Por el contrario, si $G$ se puede colorear con tres colores, nombramos los 
colores según la coloración de $M$ (que debe estar coloreado con tres colores). Debido a los triángulos 
de la figura 11.4, los colores de las variables corresponden a una asignación de verdad consistente. 
La construcción de la figura 11.6 garantiza que al menos una variable en cada cláusula esté coloreada 
con $T$. Finalmente, $G$ puede construirse claramente en tiempo polinomial, lo que completa la demostración.