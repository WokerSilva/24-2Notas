\section{Introducción}

\subsection{Definición de la IA}

Podemos partir de una sencilla definición sobre la IA: "la habilidad de los ordenadores para hacer
actividades que normalmente requieren inteligencia humana".\\

La inteligencia artificial se define como la capacidad de las máquinas para utilizar algoritmos, 
aprender de datos y aplicar ese aprendizaje en la toma de decisiones de manera similar a los humanos. A diferencia de las 
personas, los dispositivos basados en IA no requieren descanso y pueden procesar grandes cantidades 
de información simultáneamente. Además, la tasa de errores en las máquinas que realizan tareas 
comparables a las humanas es considerablemente menor.\cite{libroDos}\\

A medida que la tecnología avanza han surgido cuatro ejes para poder comprender la definición de 
inteligencia artificial donde existe un enfrentamiento entre los enfoques centrados en los humanos y los
centrados en torno a la racionalidad

\begin{myitemize}
    \item \textit{Enfoque centrado en el comportamiento humano}
    
    Este enfoque se basa en la observación y el análisis del comportamiento humano para desarrollar 
    sistemas de IA que imiten las habilidades cognitivas y físicas de las personas. Se trata de 
    un enfoque empírico que utiliza hipótesis y experimentos para confirmarlas, y se centra en la 
    inteligencia artificial débil, que se limita a realizar tareas específicas.

    \item \textit{Enfoque racional} 
    
    Este enfoque se basa en la lógica y las matemáticas para desarrollar 
    sistemas de IA que razonen y tomen decisiones de forma similar a como lo hacen los humanos. 
    Es un enfoque más teórico que se centra en la construcción de modelos simbólicos del mundo, 
    y se centra en la inteligencia artificial fuerte, que es capaz de comprender 
    y razonar como un humano.

    \item \textit{Enfoque computacional} 
    
    Este enfoque se centra en el desarrollo de algoritmos y técnicas de computación que permitan a 
    los sistemas de IA procesar información y realizar tareas de forma eficiente. Este enfoque ha 
    sido fundamental para el desarrollo de las tecnologías de IA modernas, como el 
    aprendizaje automático y el aprendizaje profundo.

    \item \textit{Enfoque en el entorno} 
    
    Este enfoque se centra en el papel del entorno en la inteligencia y el comportamiento. Este 
    enfoque reconoce que los sistemas de IA no operan en el vacío, sino que están integrados en 
    un entorno que puede influir en su comportamiento de diversas maneras.
\end{myitemize}

\begin{center}
    \includegraphics[scale = .7]{IMA/defIA.png}    
\end{center}


El enfoque centrado en el comportamiento humano se apoya en la observación y análisis del comportamiento 
humano para modelar sistemas de IA que imiten habilidades específicas. Esta perspectiva se basa en la 
recopilación de datos empíricos y la confirmación de hipótesis a través de experimentos. Sin embargo, 
su alcance se limita a tareas concretas, lo que se conoce como inteligencia artificial débil.\\


Por otro lado, el enfoque racional se adentra en la lógica y las matemáticas para crear sistemas que 
puedan razonar y tomar decisiones como lo hacen los humanos. Se centra en construir modelos simbólicos del 
mundo y aspira a la inteligencia artificial fuerte, capaz de comprender y razonar de manera más amplia.\\


El enfoque computacional se destaca por su enfoque en el desarrollo de algoritmos y técnicas 
computacionales para potenciar la capacidad de procesamiento de los sistemas de IA. Este enfoque ha 
sido crucial para el avance de tecnologías como el aprendizaje automático y el aprendizaje profundo, 
permitiendo el manejo eficiente de grandes volúmenes de datos.\\


\subsection{Áreas de la IA}


\begin{myitemize}
    \item Reconocimiento de imágenes estáticas, clasificación y etiquetado: estas herramientas son útiles
    para una amplia gama de industrias.

    \item Mejoras del desempeño de la estrategia algorítmica comercial: ya ha sido implementada de diversas maneras en el sector financiero.
    
    \item Procesamiento eficiente y escalable de datos de pacientes: esto ayudará a que la atención médica sea más efectiva y eficiente. 
    
    \item Detección y clasificación de objetos: puede verse en la industria de vehículos autónomos, aunque también tiene potencial para muchos otros campos
    
    \item Distribución de contenido en las redes sociales: se trata principalmente de una herramienta de marketing utilizada en las redes sociales, pero también puede usarse para crear conciencia entre las organizaciones sin ánimo de lucro o para difundir información rápidamente como servicio público.
    
    \item Protección contra amenazas de seguridad cibernética: es una herramienta importante para los bancos y los sistemas que envían y reciben pagos en línea
\end{myitemize}



\subsection{PLANIFICACIÓN AUTOMÁTICA}

La planificación automática es un área de la inteligencia artificial que se centra en desarrollar algoritmos 
y técnicas para generar planes o secuencias de acciones que permitan alcanzar ciertos objetivos en un entorno 
dinámico y a menudo incierto. Estos algoritmos se utilizan en una variedad de aplicaciones, como sistemas 
de control de robots, logística, juegos, sistemas de gestión de recursos, entre otros\\


Un ejemplo común de planificación automática es la planificación de rutas para la entrega de paquetes o también 
puede ser que para una empresa de fabricación que produce varios productos utilizando diferentes recursos, como 
materias primas, mano de obra y maquinaria. El objetivo es maximizar la eficiencia de la producción asignando 
adecuadamente estos recursos para satisfacer la demanda de los productos, el algoritmo de planificación 
automática analizará todas estas variables y generará un plan óptimo o subóptimo


\subsection{PROCESAMIENTO DE LENGUAJE NATURAL}

Es la comunicación entre humano-máquina por medio de los recursos generados de forma natural por los humanos.\\

Procesamiento de texto: se basa en el lenguaje escrito y la comprensión sintáctica, semántica y pragmática del mismo.\\

Procesamiento de voz: utiliza las señales fonéticas y se enfoca mayormente en la traducción de audio a texto.\\

\subsection{EJEMPLO DE PLN (ENTIDADES NOMBRADAS)}

"Tokio Blues, escrito por el autor japonés Haruki Murakami, es una novela que explora temas de amor, 
pérdida y búsqueda de identidad en la sociedad contemporánea."\\

En este caso, el PLN podría identificar y clasificar las siguientes entidades nombradas:

\begin{itemize}
    \item "Tokio Blues": sería clasificado como el título de un libro.
    \item "Haruki Murakami": sería clasificado como el nombre de un autor.
    \item "Japonés": sería clasificado como un adjetivo que describe la nacionalidad del autor.
    \item "Novela": sería clasificada como el género literario.
\end{itemize}

\subsection{APLICACIONES DEL LENGUAJE NATURAL}

\section{Aprendizaje automático}

El aprendizaje automático es una rama clave de la inteligencia artificial que se centra en desarrollar algoritmos 
y modelos que permitan a las máquinas aprender patrones y tomar decisiones sin una programación explícita. Utilizando 
técnicas estadísticas y de optimización, el aprendizaje automático capacita a las máquinas para mejorar su 
rendimiento a medida que se exponen a más datos. Esto tiene aplicaciones en una amplia gama de campos, como el 
análisis de datos, la predicción de tendencias, la personalización de contenido y la automatización de procesos.\\ 


Con su capacidad para extraer información útil de grandes volúmenes de datos y adaptarse a situaciones cambiantes, 
el aprendizaje automático continúa desempeñando un papel fundamental en la evolución y la expansión de 
la inteligencia artificial.