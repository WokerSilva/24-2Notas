\section{Algoritmo Genético}

% --------------------------------------------------------------------------------------/
% -------------------------------------------------------------------------------------/
% ------------------------------------------------------------------------------------/
\subsection*{¿Qué es el algoritmo Genético?}
% -------------------------------------------------------------------------------------/
% ------------------------------------------------------------------------------------/
% -----------------------------------------------------------------------------------/

Un algoritmo genético \texttt{(AG)} entonces es: una técnica de resolución de problemas 
que utiliza principios inspirados en la selección natural para solucionar problemas de 
optimización. La selección natural (evolución) como estrategia implica la creación, 
reproducción y adaptación de una población de posibles soluciones en un rango de generaciones.\\ 

Este enfoque de población se basa en que cada individuo representa una posible solución 
al problema, y la evolución de estos comienza por, la \textbf{creación} de los individuos, 
la \textbf{reproducción} para después hacer la óptima \textbf{selección} de los padres en 
una nueva reproducción y \textbf{mutación}, finalmente comprobar o evaluar su \textbf{aptitud}\\ 

% -------------------------------------------------------------------------------------/
% ------------------------------------------------------------------------------------/
\subsubsection*{Algoritmo genético vs. Búsqueda no informada}
% -------------------------------------------------------------------------------------/
% ------------------------------------------------------------------------------------/

Los algoritmos genéticos y la búsqueda no informada son enfoques diferentes para resolver 
problemas de optimización. Un algoritmo genético es una técnica inspirada en la evolución 
biológica, donde una población de soluciones potenciales evoluciona a lo largo de 
generaciones mediante operadores como la selección, el cruce y la mutación. Por ejemplo, 
imagina que queremos encontrar la mejor ruta para entregar paquetes en una ciudad. Un 
algoritmo genético podría representar diferentes rutas como individuos en una población, 
y a través de generaciones, las mejores rutas (soluciones) se \textit{reproducen} y mejoran.\\ 


En contraste, la búsqueda no informada es más como un proceso de prueba y error sin guía 
específica. Por ejemplo, si estuviéramos buscando un objeto en una habitación oscura, 
podríamos usar un enfoque de búsqueda no informada moviéndonos aleatoriamente y esperando 
encontrar el objeto eventualmente. Sin embargo, este enfoque puede ser ineficiente ya que 
no utiliza información sobre la ubicación probable del objeto.


% -------------------------------------------------------------------------------------/
% ------------------------------------------------------------------------------------/
\subsubsection*{Algoritmo genético vs. Búsqueda informada}
% -------------------------------------------------------------------------------------/
% ------------------------------------------------------------------------------------/

Los algoritmos genéticos y la búsqueda informada se diferencian en la forma en que 
utilizan la información sobre el problema para dirigir la búsqueda hacia soluciones 
óptimas. En el caso de un algoritmo genético, la información proviene de la evaluación de 
la aptitud de las soluciones en función de un objetivo específico. Por ejemplo, en un 
problema de programación de horarios escolares, las soluciones que evitan superposiciones 
de clases y maximizan el uso de recursos se considerarían más aptas.\\ 

Por otro lado, la búsqueda informada, como el algoritmo $A^{*}$ utiliza información más 
detallada sobre el problema, como funciones heurísticas que estiman el costo desde un 
estado dado hasta el objetivo. Siguiendo con el ejemplo del horario escolar, una función 
heurística podría estimar la eficiencia de una programación en términos de número de horas 
de clase y recursos utilizados. Esto permite a la búsqueda informada dirigirse hacia 
soluciones prometedoras de manera más eficiente que la búsqueda no informada o los 
algoritmos genéticos, especialmente en problemas donde se dispone de información detallada 
sobre el dominio del problema.



% -------------------------------------------------------------------------------------/
% ------------------------------------------------------------------------------------/
% -----------------------------------------------------------------------------------/
\subsection{Pasos que sigue}
% -------------------------------------------------------------------------------------/
% ------------------------------------------------------------------------------------/
% -----------------------------------------------------------------------------------/


% ------------------------------------------------------------------------------------/
% -----------------------------------------------------------------------------------/
\subsubsection*{Inicialización de la Población 01}
% -------------------------------------------------------------------------------------/
% ------------------------------------------------------------------------------------/

Comenzamos con una población inicial de posibles rutas para la entrega de paquetes. 
Cada individuo en la población representa una ruta diferente que conecta diferentes 
ubicaciones en la ciudad.

% ------------------------------------------------------------------------------------/
% -----------------------------------------------------------------------------------/
\subsubsection*{Evaluación de la Aptitud 02}
% -------------------------------------------------------------------------------------/
% ------------------------------------------------------------------------------------/

Se evalúa cada ruta utilizando una función de aptitud que considera factores como la 
distancia recorrida, el tiempo de entrega y la eficiencia en el uso de recursos. 
Esto asigna a cada ruta un valor numérico que indica qué tan buena es en comparación 
con otras.


% ------------------------------------------------------------------------------------/
% -----------------------------------------------------------------------------------/
\subsubsection*{Selección de Padres 03}
% -------------------------------------------------------------------------------------/
% ------------------------------------------------------------------------------------/

Los individuos con mayor aptitud tienen una mayor probabilidad de ser seleccionados 
como padres para la reproducción. Esto se hace de manera aleatoria, pero con una mayor 
probabilidad para los individuos más aptos.

% ------------------------------------------------------------------------------------/
% -----------------------------------------------------------------------------------/
\subsubsection*{Reproducción 04} 
% -------------------------------------------------------------------------------------/
% ------------------------------------------------------------------------------------/

Se eligen dos padres de la población seleccionada y se combinan para crear nuevos 
descendientes. Esto se logra cruzando las cadenas genéticas de los padres en un punto 
de cruce aleatorio. Por ejemplo, si un padre tiene una ruta que pasa por un conjunto 
de ubicaciones y el otro padre tiene otra ruta que pasa por diferentes ubicaciones, 
el cruce en un punto específico podría combinar partes de ambas rutas para generar 
una nueva.


% ------------------------------------------------------------------------------------/
% -----------------------------------------------------------------------------------/
\subsubsection*{Mutación 05}
% -------------------------------------------------------------------------------------/
% ------------------------------------------------------------------------------------/

Los nuevos descendientes pueden experimentar cambios aleatorios en sus rutas, simulando 
la mutación genética. Esto introduce variabilidad en la población y evita la convergencia 
prematura hacia una solución subóptima.


% ------------------------------------------------------------------------------------/
% -----------------------------------------------------------------------------------/
\subsubsection*{Reemplazo 06}
% -------------------------------------------------------------------------------------/
% ------------------------------------------------------------------------------------/

Los nuevos descendientes reemplazan a los individuos menos aptos en la población 
existente. Esto ayuda a mantener la diversidad genética en la población y a explorar 
diferentes soluciones potenciales.

% ------------------------------------------------------------------------------------/
% -----------------------------------------------------------------------------------/
\subsubsection*{Criterio de Aptitud 07}
% -------------------------------------------------------------------------------------/
% ------------------------------------------------------------------------------------/

Se repiten los pasos 2 a 6 durante varias generaciones hasta que se alcance algún 
criterio de aptitud, como una solución que cumpla con los requisitos de entrega de paquetes
(por ejemplo, la ruta más corta que visita todas las ubicaciones requeridas) o un número 
máximo de generaciones.
