\section{Búsqueda}

La búsqueda juega un papel fundamental en el corazón de la inteligencia artificial. Desde la planificación
de rutas óptimas en sistemas de navegación hasta la toma de decisiones en entornos complejos, 
los algoritmos de búsqueda son la columna vertebral de muchas aplicaciones de IA.\\ 

% -------------------------------------------------------------------------------------/
% ------------------------------------------------------------------------------------/
% -----------------------------------------------------------------------------------/
\subsection{Resolver problemas mediante búsquedas}
% -------------------------------------------------------------------------------------/
% ------------------------------------------------------------------------------------/
% -----------------------------------------------------------------------------------/

% -------------------------------------------------------------------------------------/
% ------------------------------------------------------------------------------------/
\subsubsection*{Agentes resolventes-problemas}
% -------------------------------------------------------------------------------------/
% ------------------------------------------------------------------------------------/

Los agentes resolventes-problemas deciden qué hacer para encontrar secuencias de acciones que conduzcan 
a los estados deseables. Comenzamos definiendo con precisión los elementos que constituyen el 
\textit{problema} y su \textit{solución} y daremos diferentes ejemplos para ilustrar estas 
definiciones. Entonces, describimos diferentes algoritmos de propósito general que podamos utilizar 
para resolver estos problemas y así comparar las ventajas de cada algoritmo. Los algoritmos son no 
informados, en el sentido que no dan información sobre el problema salvo su definición.\\ 


Los agentes solucionadores de problemas son capaces de representar el problema en términos formales, 
identificar y evaluar posibles acciones, y seleccionar las más prometedoras en función de ciertos 
criterios, como la optimización del costo o la eficiencia. Además, pueden utilizar información adicional, 
como heurísticas, para guiar la búsqueda hacia soluciones más rápidas y efectivas.\\ 

% -------------------------------------------------------------------------------------/
% ------------------------------------------------------------------------------------/
\subsubsection*{Problemas y Soluciones}
% -------------------------------------------------------------------------------------/
% ------------------------------------------------------------------------------------/

En el contexto de la inteligencia artificial, definir un problema de manera formal es esencial para 
aplicar algoritmos de búsqueda y encontrar soluciones efectivas. Un problema formal consta de 
varios componentes fundamentales que permiten al agente solucionador de problemas entender la 
naturaleza del problema y trabajar hacia su resolución de manera sistemática.\\ 

Para ilustrar estos componentes, consideremos el problema de planificación de ruta en un mapa. 
Imaginemos que tenemos un mapa de una ciudad y deseas encontrar la ruta más corta desde tu 
ubicación actual hasta un destino específico.

\begin{myitemize}
    \item Estado Inicial: El estado inicial en este problema sería la ubicación actual en el mapa.
        
    \item Descripción de las Posibles Acciones Disponibles: Las posibles acciones disponibles son las 
    decisiones que el agente puede tomar en cada estado para avanzar hacia el objetivo. Estas acciones 
    podrían incluir moverse hacia adelante, girar a la derecha, girar a la izquierda o retroceder. 
    La formulación más común utiliza una función sucesor para describir estas acciones en función del 
    estado actual del agente.
    
    \item Test Objetivo: El test objetivo define las condiciones que deben cumplirse para determinar si 
    el objetivo ha sido alcanzado. En nuestro ejemplo, el test objetivo sería llegar a la ubicación 
    deseada en el mapa.
    
    \item Costo del Camino: El costo del camino es la medida del esfuerzo o la distancia asociada con 
    cada acción tomada por el agente. En el contexto de la planificación de ruta, el costo del camino 
    podría ser la distancia recorrida desde el estado inicial hasta el estado objetivo.
\end{myitemize}


El problema del 8-Puzle es un clásico problema de rompecabezas deslizante en el que el objetivo es 
organizar una cuadrícula de 3x3 con fichas numeradas del 1 al 8, más un espacio en blanco, en un 
estado objetivo particular.

\begin{myitemize}
    \item \textbf{Estado Inicial:} Tenemos un tablero inicial de 3x3, con 8 fichas que se mueven las 
    cuales están distribuidas de manera aleatoria dentro del tablero.

    \item \textbf{Acciones Disponibles:} Las acciones disponibles para mover las fichas son: arriba,
    abajo, derecha o izquierda y lo movimientos solo pasan si es a través de cuadro blanco o vacio.         
    
    \item \textbf{Test Objetivo:} Se compara el estado actual del rompecabezas con el estado objetivo 
    El cual es que las fichas estén ordenadas desde 1 a 8 y  el espacio en blanco está en la 
    ubicación deseada.
    
    \item \textbf{Costo del Camino:} El costo del camino se calcula contando el número de movimientos 
    realizados desde el estado inicial hasta el estado objetivo. Cada movimiento de una ficha se suma al 
    costo total del camino.
\end{myitemize}


% -------------------------------------------------------------------------------------/
% ------------------------------------------------------------------------------------/
\subsubsection*{Elementos de los problemas de búsqueda}
% -------------------------------------------------------------------------------------/
% ------------------------------------------------------------------------------------/

\begin{myitemize}
    \item Árbol de Búsqueda: Se puede representar la exploración del espacio de estados
            como un árbol de búsqueda, donde cada nodo es un estado y las aristas son las
            acciones que conducen de un estado a otro.

    \item Complejidad Temporal y Espacial: La complejidad temporal se refiere al tiempo
            necesario para encontrar una solución, mientras que la complejidad espacial se
            refiere al espacio de almacenamiento necesario durante la búsqueda.

    \item Búsqueda No Informada (ciega): En la búsqueda no informada, el agente no tiene
            información sobre la probabilidad de alcanzar el estado objetivo desde un estado
            dado. Métodos como la búsqueda en anchura y la búsqueda en profundidad son
            ejemplos de estrategias no informadas.

    \item Búsqueda Informada (heurística): En la búsqueda informada, el agente utiliza
            información adicional para guiar la exploración hacia el estado objetivo de manera
            más eficiente. La heurística es una función que estima el costo desde un estado
            hasta el objetivo.            

    \item Heurísticas Admisibles e Inadmisibles: Una heurística es admisible si nunca
            sobreestima el costo para llegar al objetivo desde un estado dado. Las
            heurísticas inadmisibles pueden sobreestimar o subestimar el costo.

    \item Optimalidad: Una solución es óptima si es la más corta o la de menor costo
            entre todas las soluciones posibles.
            
    \item Problema del Espacio de Estado Infinito: Algunos problemas de búsqueda
            pueden tener un espacio de estados infinito, lo que plantea desafíos
            adicionales en la representación y exploración.
\end{myitemize}


\subsubsection*{Árbol de búsqueda}

\noindent Vamos a ilustrar una solución de búsqueda por árbol para el problema del 8-Puzle,
modelamos el problema y aplicaremos un algoritmo de búsqueda no informada.\\ 

\noindent El planteamiento del Problema: El objetivo es organizar las fichas numeradas
del 1 al 8 en una cuadrícula de 3x3, dejando el espacio en blanco en la posición inferior derecha.

\noindent Modelado del Problema:

\begin{myitemize} 
    \item Estado Inicial: Un tablero de 3x3 con las fichas distribuidas aleatoriamente.
    \item Acciones Disponibles: Mover una ficha adyacente al espacio en blanco hacia arriba, abajo, 
            derecha o izquierda.
    \item Test Objetivo: El tablero coincide con el estado objetivo.
    \item Costo del Camino: Cada movimiento incrementa el costo en 1.
\end{myitemize}

Árbol de Búsqueda: Representamos cada estado del tablero como un nodo y cada movimiento posible como una 
arista que conecta los nodos.
\begin{myitemize} 
    \item[] Complejidad
    \begin{myitemize} 
        \item Temporal: El tiempo necesario para encontrar una solución puede ser exponencial, ya que el 
                número de estados posibles es muy grande.
        \item Espacial: El espacio necesario para almacenar el árbol de búsqueda también puede 
                ser exponencial.
    \end{myitemize}

    \item[] Búsqueda No Informada: Utilizaremos la búsqueda en anchura (BFS) Proceso de Búsqueda:
    \begin{myitemize} 
        \item Generamos el nodo raíz con el estado inicial.
        \item Expandimos el nodo raíz creando nodos hijos para cada acción posible.
        \item Continuamos expandiendo los nodos en orden de anchura, nivel por nivel.
        \item Si un nodo alcanza el test objetivo, detenemos la búsqueda.
    \end{myitemize}    

    \item[] Optimalidad: La búsqueda en anchura garantiza encontrar la solución óptima en términos de 
            menor número de movimientos.
\end{myitemize}


% -------------------------------------------------------------------------------------/
% ------------------------------------------------------------------------------------/
% -----------------------------------------------------------------------------------/
\subsection{Busqueda No Informada}
% -------------------------------------------------------------------------------------/
% ------------------------------------------------------------------------------------/
% -----------------------------------------------------------------------------------/

Imaginemos que estamos por comenzar a caminar en un bosque sin embargo lo haremos sin un mapa. La 
estrategia sería explorar y probar caminos hasta encontrar la salida. Esto es en palabras sencillas 
lo que hace una busqueda no informada: exploran sin pistas adicionales, solo la definición del 
problema. Este es un enfoque utilizado en inteligencia artificial para resolver problemas donde no 
se dispone de información adicional sobre el espacio de búsqueda se basa únicamente en la 
exploración sistemática del espacio de estados, sin utilizar heurísticas o conocimiento específico 
sobre el problema.\\ 

\subsubsection*{Búsqueda en primero a lo ancho (BFS)} 

Es como una ola que se expande en el agua. Desde tu punto de partida, revisas todos los caminos 
cercanos antes de ir más lejos. Es opción interesante para encontrar la ruta más corta en un mapa 
o el menor número de movimientos en un juego de mesa.

\begin{myitemize} 
    \item[] Pseudocódigo BFS
    \begin{verbatim}
    BFS(estado_inicial, test_objetivo):
        crea una cola con el estado_inicial
        mientras la cola no esté vacía:
            estado = cola.desencolar()
            si test_objetivo(estado):
                devuelve la solución
        para cada acción en estado:
                hijo = aplica acción a estado
                cola.encolar(hijo)
        devuelve fracaso
    \end{verbatim}
\end{myitemize}

\subsubsection*{Búsqueda primero en lo profundo (DFS)} 

Es como seguir un camino en el bosque hasta el final antes de regresar y probar otro. Es útil 
cuando tienes muchos caminos y no te importa cuán largo sea, solo quieres llegar al 
final rápidamente.


\begin{myitemize} 
    \item[] Pseudocódigo DFS:
    \begin{verbatim}
    DFS(estado_inicial, test_objetivo):
        crea una pila con el estado_inicial
        mientras la pila no esté vacía:
            estado = pila.desapilar()
            si test_objetivo(estado):
                devuelve la solución
            para cada acción en estado:
                hijo = aplica acción a estado
                pila.apilar(hijo)
        devuelve fracaso    
    \end{verbatim}
\end{myitemize}


Búsqueda de costo uniforme: Piensa en un sistema de subasta donde cada movimiento 
tiene un precio y quieres gastar lo menos posible. Esta búsqueda siempre elige el 
camino más barato hasta ahora, ideal para planificar presupuestos o rutas de viaje 
con diferentes costos.\\ 

Búsqueda en profundidad limitada (DLS): Es como DFS, pero con un límite. Te pones 
un casco con una linterna que solo alcanza cierta distancia. Exploras hasta ese 
límite y si no encuentras la salida, retrocedes. Es bueno para cuando DFS se va 
demasiado profundo y quieres ponerle un toque de control.

% -------------------------------------------------------------------------------------/
% ------------------------------------------------------------------------------------/
% -----------------------------------------------------------------------------------/
\subsection{Busqueda Informada}
% -------------------------------------------------------------------------------------/
% ------------------------------------------------------------------------------------/
% -----------------------------------------------------------------------------------/

\subsubsection*{Funciones heurísticas}

Las funciones heurísticas son como brújulas que orientan a los algoritmos de búsqueda 
hacia la solución. Son métodos de evaluación que proporcionan una estimación del costo 
o la distancia desde un estado dado hasta el objetivo en un problema de búsqueda. Estas 
funciones son fundamentales en las estrategias de búsqueda informada, ya que guían la 
exploración del espacio de búsqueda al proporcionar información sobre qué tan cerca está  
un estado del objetivo. Una buena función heurística es aquella que es admisible (no 
sobreestima el costo) y consistente (sigue un patrón predecible de valores). 

\subsubsection*{Búsqueda voraz primero el mejor}

La búsqueda voraz primero el mejor es una estrategia que elige el siguiente nodo a 
explorar basándose únicamente en el costo estimado desde el nodo actual hasta el objetivo. 
No considera el costo total acumulado hasta el momento, lo que puede llevar a soluciones 
rápidas pero no siempre óptimas. Es como decidir qué camino tomar en un laberinto mirando 
solo lo que está más cerca del final, sin pensar en si ese camino podría tener obstáculos 
más adelante.

\subsubsection*{Búsqueda $A^{*}$}

La búsqueda $A^{*}$ es una estrategia de búsqueda informada que combina la ventaja de la 
búsqueda voraz primero el mejor (evaluando nodos basándose en la estimación del costo 
hasta el objetivo) con la consideración del costo total acumulado hasta el momento. Esto 
permite encontrar una solución óptima si la función heurística utilizada es admisible y 
consistente. Es como tener un mapa detallado del laberinto y usarlo para tomar decisiones 
inteligentes sobre qué camino tomar, considerando tanto la distancia al final como los 
obstáculos en el camino.

\subsubsection*{Búsqueda heurística con memoria acotada}

La búsqueda heurística con memoria acotada es una estrategia que utiliza información 
parcial del espacio de búsqueda para limitar la exploración a áreas más prometedoras. 
Esto se logra manteniendo una lista de nodos prometedores conocidos como 
\textit{memoria acotada}, lo que evita la repetición de exploraciones innecesarias y 
mejora la eficiencia de la búsqueda.